\newglossaryentry{System}{name={System},
	description={Raspberry Pi 3b mit BCM 2837 (vier Prozessorkernen) und einem GibiByte Hauptspeicher}}

\newglossaryentry{c}{name={Cache},
    description={hier Hardwarecache: kleiner, schneller und für den Prozessor transparenter Pufferspeicher zwischen Prozessor und Hauptspeicher für Daten oder Instruktionen}}

\newglossaryentry{wcet}{name={WCET-Analyse},
    description={Worst-Case-Execution-Time-Analysis: Menge analytischer Methoden zur Ermittlung einer oberen Schranke für die Ausführungszeit von Programmcode}}

\newglossaryentry{mw}{name={Memory-Wall},
    description={Die Zugriffsgeschwindigkeit des Hauptspeichers ist deutlich geringer, als die Rechengeschwindigkeit von Prozessoren}}

\newglossaryentry{tr}{name={Trashing},
	description={Wechselseitige Verdrängung von Cachelines aus einem Cache}}

\newglossaryentry{zl}{name={zeitliche Lokalität},
	description={Auf Daten, auf die schon einmal zugegriffen wurde, wird mit hoher Wahrscheinlichkeit erneut zugegriffen}}

\newglossaryentry{rl}{name={räumliche Lokalität},
	description={Es wird mit hoher Wahrscheinlichkeit auf Daten zugegriffen, wenn bereits auf Daten mit benachbarten Adressen zugegriffen wurde}}

\newglossaryentry{pre}{name={Prefetching},
	description={Laden von Daten von langsamen in schnellere Ebenen der Speicherhierarchie vor ihrer Referenzierung}}

\newglossaryentry{ccr}{name={Cycle Count Register},
	description={Register \texttt{PMCCNTR\_EL0} der Performance Monitor Unit des Systems, wird mit jedem Zyklus inkrementiert}}
	
\newglossaryentry{gcc}{name={Compiler},
	description={hier: C-Compiler der GNU Compiler Collection}}
	
