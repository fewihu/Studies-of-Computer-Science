\documentclass{beamer}

\usepackage{tikz}
\usetikzlibrary{decorations.pathreplacing}
\usepackage{pgfplots}
\pgfplotsset{compat=1.5,}

\usepackage[utf8]{inputenc}
\usepackage[T1]{fontenc}

\usepackage[ngerman]{babel}
\usepackage{csquotes}
\MakeOuterQuote{"}

\usepackage{xcolor}

\usepackage{listings}
\definecolor{codegreen}{rgb}{0,0.6,0}
\definecolor{codegray}{rgb}{0.5,0.5,0.5}
\definecolor{codepurple}{rgb}{0.58,0,0.82}
\definecolor{backcolour}{rgb}{0.98,0.99,0.97}
\lstdefinestyle{mystyle}{
    backgroundcolor=\color{backcolour},   
    commentstyle=\color{codegreen},
    keywordstyle=\color{magenta},
    numberstyle=\tiny\color{codegray},
    stringstyle=\color{codepurple},
    basicstyle=\ttfamily,
    breakatwhitespace=false,         
    breaklines=true,                 
    captionpos=b,                    
    keepspaces=true,                 
    numbers=left,                    
    numbersep=5pt,                  
    showspaces=false,                
    showstringspaces=false,
    showtabs=false,                  
    tabsize=2
}
\lstset{style=mystyle}
\lstset{escapeinside={<@}{@>}}

\title{Verteidigung der Bachelorarbeit}
\subtitle{Ermittlung von Cacheeigenschaften für
ARM-Prozessorkerne}
\author{Felix Müller}
\date{\today}
\usetheme{Boadilla}
\usecolortheme{crane}
\setbeamertemplate{navigation symbols}{}

% Titel
\begin{document}
\begin{frame}
\titlepage
\end{frame}

% TOC
\begin{frame}{Inhaltsübersicht}
\tableofcontents
\end{frame}

\section{Einleitung}
\begin{frame}
	Ziel: Ersetzungsstrategie des Level-1-Caches des Arm Cortex A53 experimentell bestimmen
\end{frame}

\begin{frame}
	
\end{frame}

\section{Grundlagen}
\begin{frame}
	\frametitle{Inhaltsübersicht}
	\tableofcontents[currentsection]
\end{frame}

\subsection{Caches}
\begin{frame}
	
\end{frame}

\subsection{WCET-Analyse}
\begin{frame}
	
\end{frame}

\subsection{Vorarbeiten}
\begin{frame}
	
\end{frame}

\section{Latenzmessung}
\begin{frame}
	\frametitle{Inhaltsübersicht}
	\tableofcontents[currentsection]
\end{frame}

\subsection{Cycle Count Register}
\begin{frame}
	\frametitle{Cycle Count Register}
	\begin{itemize}
		\item Performance Monitor Unit bietet Register \texttt{PMCCNTR\_EL0}
		\item analog zu \textit{Time Stamp Counter} der \texttt{x86}\,-Familie
		\item Zugriff muss aktiviert werden\footnote{\tiny Arm. Arm Cortex -A53 MPCore Processor Technical Reference Manual. 2014. S.71}
		\item \texttt{mrs <register>, pmccntr\_el0}
	\end{itemize}
\end{frame}

\begin{frame}[fragile]
\frametitle{Code Latenzmessung}
\begin{figure}
\centering
\begin{lstlisting}[language=C]
asm volatile("mrs %0, pmccntr_el0" : "=r"(start));
zu vermessende LOAD-Instruktion
asm volatile("mrs %0, pmccntr_el0" : "=r"(dur));
dur -= start;
\end{lstlisting}
\caption{C-Code zum zweimaligen Auslesen des \textsl{CCR}}
\end{figure}
\end{frame}

\begin{frame}[fragile]
\frametitle{Code Latenzmessung 2}
\begin{figure}
\begin{lstlisting}[language=C]
isb();
asm volatile("mrs %0, pmccntr\el0" : "=r"(start));
zu vermessende LOAD-Instruktion
asm volatile("dmb ld");
asm volatile("mrs %0, pmccntr_el0" : "=r"(dur));
dur -= start;
\end{lstlisting}
\caption{C-Code zum zweimaligen Auslesen des \textsl{CCR}}
\end{figure}


\begin{enumerate}
	\item \texttt{isb}
	\item \texttt{dmb ld}
\end{enumerate}

\end{frame}

\subsection{Beinflussung der Messung}
\begin{frame}
\frametitle{Beinflussung der Messung}
\begin{columns}[c]
\column {0.5\textwidth}
\begin{itemize}
	\item Instruction Reordering
	\item Dual Issue Pipeline
	\item Unterbrechung
	\item nicht-blockierender Cache
\end{itemize}
\column {0.5\textwidth}
\end{columns}
\end{frame}

\begin{frame}
\frametitle{Beinflussung der Messung}
\begin{columns}[c]
\column {0.45\textwidth}
\begin{itemize}
	\item Instruction Reordering
	\item Dual Issue Pipeline
	\item Unterbrechung
	\item nicht-blockierender Cache
\end{itemize}
\column {0.55\textwidth}
\begin{itemize}
	\item[$\rightarrow$] \textcolor{black!50!green}{In-Order Pipeline}
	\item[$\rightarrow$] \textcolor{black!50!green}{Instruction Synchronization Barrier}
	\item[$\rightarrow$] \textcolor{black!50!green}{Unterbrechung verhindern}
	\item[$\rightarrow$] \textcolor{black!50!green}{Date Memory Barrier}
\end{itemize}
\end{columns}
\end{frame}

\begin{frame}
\frametitle{Beinflussung der Messung: Instruction Synchronization Barrier}
\begin{itemize}
	\item \textit{ISB} führt zum Leeren der \textit{Pipeline}
	\item dadurch Verhalten bzgl. \textit{Dual-Issue} identisch
	\item Messwerte sind vergleichbar und geeignet Cachezustand zu bestimmen
\end{itemize}
\end{frame}

\begin{frame}
\frametitle{Beinflussung der Messung: Unterbrechung verhindern}
\begin{itemize}
	\item Implementierung als Kernelmodul
	\item Spinlock mit Interruptsperre
\end{itemize}
\end{frame}

\begin{frame}
\frametitle{Beinflussung der Messung: Data Memory Barrier}
\begin{itemize}
	\item Zugriffslatenzen sollen kaschiert werden
	\item Abarbeitung von Instruktionen, die semantisch nicht von Daten vorheriger \texttt{load}-Instruktionen abhängen
	\item Verhalten des Arm Cortex A53 nicht bekannt
	\item \textit{Data Memory Barrier} blockiert Abarbeitung bis Speicheroperartion abgeschlossen
\end{itemize}
\end{frame}

\subsection{Inferenz des Cachszustands}
\begin{frame}
\frametitle{Inferenz des Cachszustands}
\begin{itemize}
	\item Messung dauert \textbf{4 Zyklen ohne Zugriff}
	\item Zugriffslatenz u.\,d.\,B. Cache-Hit: \textbf{3 Zyklen}\footnote{\tiny Arm. Arm Cortex -A53 MPCore Processor Technical Reference Manual. 2014. S.18.}
	\item ab $8$ Zyklen ist von Cache-Miss auszugehen
	\item \textbf{Heuristik} $f_{\textrm{Hit}}: \mathbb{N} \rightarrow \mathbb{B}$
	\begin{equation}
	f_{\textrm{Hit}} (t) =
	\begin{cases}
        0 & \textrm{wenn } t > 7\\
        1 & \textrm{sonst}
	\end{cases} 
\end{equation}
\end{itemize}
\end{frame}

\section{Bestimmung der Kapazität}
\begin{frame}
	\frametitle{Inhaltsübersicht}
	\tableofcontents[currentsection]
\end{frame}

\subsection{Idee}
\begin{frame}
\frametitle{Idee zur Bestimmung der Kapazität}
\begin{itemize}
	\item Level-1-Datencache kann Werte \textbf{8, 16, 32 oder 64 $\textrm{kiB}$} haben
	\item Zugriffssequenz aufsteigender Adressen $M_0 \dots M_a$
	\item erstmalige Durchführung: \textbf{Cache wird gefüllt}
	\item nochmalige Durchführung mit \textbf{Vermessung}
\end{itemize}
\end{frame}

\subsection{Implementierung}
\begin{frame}
\frametitle{Implementierung}
\begin{itemize}
	\item Level-1-Datencache ist \textbf{physisch adressiert}
	\item Zugriffe müssen auf physisch aufsteigende Adressen erfolgen
	\item außreichende Menge \textbf{physisch zusammenhängender Speicher}
	\item \texttt{\_\_get\_free\_pages (gfp\_t gfp\_mask, unsigned int order)} alloziert $2^{\textrm{order}}$ Hauptspeicherseiten (à $4096 \,\textrm{kiB}$)
	\item [\textbf{$\rightarrow$}] \textbf{Kernelmodul}
\end{itemize}
\end{frame}

\begin{frame}
\frametitle{Implementierung 2}
\begin{itemize}
	\item Prefetcher erkennt Sequenzen mit Versatz bis $256 \,\textrm{kiB}$
	\item Zugriffe des Programms fallen darunter
	\item Messung könnte durch Prefetching verfälscht werden
	\item automatisches Prefetching deaktivieren\footnote{\tiny Arm. Arm Cortex -A53 MPCore Processor Technical Reference Manual. 2014. S.180.}
\end{itemize}
\end{frame}

\subsection{Ergebnisse}
\begin{frame}
\begin{figure}[h]
\begin{tikzpicture}[scale=0.67]

\pgfplotsset{
    width=16cm,
    xmin=0.5,
    xmax=40.5,
    xlabel=angenommene Cachekapazität in $\textrm{kiB}$,
    grid style=dashed,
    xmajorgrids=true,
    xminorgrids=true
}

\begin{axis}[
  axis y line*=left,
  ymin=0, ymax=0.25,
  ylabel=Anzahl Cache-Misses pro Cacheline ,
  legend pos=north west,
  ymajorgrids=true,
  yminorgrids=true
]

\addplot +[thick,mark=none,black!50!red] coordinates {(32, 0) (32, 0.25)};

\addplot[very thick,mark=x,blue,only marks]
  coordinates{
    (1,0 ) (2,0 ) (3,0 ) (4,0 ) (5,0 ) (6,0 ) (7,0 ) (8,0 ) (9,0 )
    (10,0 )(11,0 )(12,0 )(13,0 )(14,0 )(15,0 )(16,0 )
    (17,0.0037)		(18,0.0069)	(19,0.0066)
    (20,0.0031)		(21,0.0030)	(22,0.0057)
    (23,0.0082)		(24,0.0156)	(25,0.0200)
    (26,0.0216)		(27,0.0255)	(28,0.0335)
    (29,0.0366)		(30,0.0458)	(31,0.0403)
    (32,0.0469)		(33,0.0758)	(34,0.1048)
    (35,0.1054)		(36,0.1406)	(37,0.1351)
    (38,0.1793)		(39,0.2308)	(40,0.2438)
};\legend{Cache-Misses pro Cacheline}
\end{axis}
\end{tikzpicture}
\end{figure}

\end{frame}

\section{Bestimmung der Ersetzungsstrategie}
\begin{frame}
	
\end{frame}

\section{Zusammenfassung \& Ausblick}
\begin{frame}
	
\end{frame}

\end{document}
